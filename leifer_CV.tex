% Andrew Leifer's CV
% Based on Joshua Shaevitz's Latex CV template of unknown origin

%%%%%%%%%%%%%%%%%%%%%%%%%%%% Document Setup %%%%%%%%%%%%%
\documentclass[11pt]{article}
\usepackage[paper=letterpaper,margin=1in]{geometry}

%% More layout: Get rid of indenting throughout entire document
\setlength{\parindent}{0in}

%% This gives us fun enumeration environments. compactitem will be nice.
\usepackage{paralist}

% This is a helpful package that puts math inside length specifications
\usepackage{calc}

% Simpler bibsection for CV sections
% (thanks to natbib for inspiration)
\makeatletter
\newlength{\bibhang}
\setlength{\bibhang}{2em}
\newlength{\bibsep}
\newcounter{Lcount}
 {\@listi \global\bibsep\itemsep \global\advance\bibsep by\parsep}
\newenvironment{bibsection}%
        {\begin{list}{\arabic{Lcount}.}{%
       \usecounter{Lcount}%
       \setlength\labelwidth{-0.5em}%
       \setlength{\leftmargin}{\bibhang}%
       \setlength{\itemindent}{-\leftmargin}%
       \setlength{\itemsep}{\bibsep}%
       \setlength{\parsep}{\z@}%
        \setlength{\partopsep}{0pt}%
        \setlength{\topsep}{0pt}}}
        {\end{list}\vspace{-.6\baselineskip}}
\makeatother

\usepackage{lastpage} 
\usepackage{fancyhdr}
\usepackage{hyperref}

\pagestyle{fancy}
%\pagestyle{empty}      % Uncomment this to get rid of page numbers
\fancyhf{}\renewcommand{\headrulewidth}{0pt}
\fancyfootoffset{\marginparsep+\marginparwidth}
\newlength{\footpageshift}
\setlength{\footpageshift}
          {0.5\textwidth+0.5\marginparsep+0.5\marginparwidth-2in}
%\lfoot{\hspace{\footpageshift}%
%       \parbox{4in}{\, \hfill %
%                    \arabic{page} of \protect\pageref{LastPage} % +LP
%                    \arabic{page}                               % -LP
%                    \hfill \,}}
\lhead{Andrew~M.~Leifer}
\rhead{Curriculum Vitae}
\fancyfoot[CO,CE]{ \arabic{page} of \protect\pageref{LastPage}}
\renewcommand{\headrulewidth}{.1pt}


%%%%%%%%%%%%%%%%
%Lists and Section headings
%%%%%%%%%%%%%%%
\newenvironment{list1}{
%\vspace{-\baselineskip}
  \begin{list}{}{
      \setlength{\itemsep}{0in}
      \setlength{\parsep}{0in} \setlength{\parskip}{0in}
      \setlength{\topsep}{0in} \setlength{\partopsep}{0in} 
      \setlength{\leftmargin}{0.2in}\setlength{\rightmargin}{0.2in}}}{\end{list}\vspace{\baselineskip}}

\newenvironment{listinline}{
	%\vspace{-\baselineskip}
	  \begin{list}{}{
	      \setlength{\itemsep}{0in}
	      \setlength{\parsep}{0in} \setlength{\parskip}{0in}
	      \setlength{\topsep}{0in} \setlength{\partopsep}{0in} 
	      \setlength{\leftmargin}{0.2in}\setlength{\rightmargin}{0.2in}}}{\end{list}}


\makeatletter
\renewcommand{\section}{\@startsection
{section}%                   % the name
{1}%                         % the level
{0mm}%                       % the indent
{2\baselineskip}%            % the before skip
{0.5\baselineskip}%          % the after skip
{\normalfont\bf\MakeUppercase}} % the style
\makeatother
\setcounter{secnumdepth}{-1} 

%%%%%%%%%%%%%%%%%%%%%%%% End Document Setup %%%%%%%%%%%%%%%%%%%%%%%%%%%%

\begin{document}
\thispagestyle{empty}

\begin{center}
\vspace{-\baselineskip}
{\large {\bf CURRICULUM VITAE
\vskip 0.1in
Andrew M.~Leifer}

Associate Professor of Physics and Neuroscience}
\end{center}

%%%%%%%%%%%%%%%%%%%%%%%%%
\section{Contact Information}

Joseph Henry Laboratories  \hfill Phone: (609) 258-8779 \\
Princeton University \hfill leifer@princeton.edu\\
Princeton, NJ 08544 \hfill \url{http://leiferlab.princeton.edu}\\

%%%%%%%%%%%%%%%%%%%%%%%%%

\vspace{-\baselineskip}
\section{Professional Experience}

{\bf Princeton University}, Princeton, NJ  \dotfill 2024--present\\
{\it Associate Professor}, Department of Physics and Princeton
Neuroscience Institute \\
%{\it Faculty}, Center for Physics of Biological Function\\ %sine 2016
{\it Associated Faculty},  Omenn-Darling Bioengineering Institute (2021-present) \\ %since 2021
{\it Associated Faculty}, Lewis-Sigler Institute for Integrative Genomics  (2016-present) \\ % since 2016
\\
{\bf Princeton University}, Princeton, NJ  \dotfill 2016--2024\\
{\it Assistant Professor}, Department of Physics and Princeton
Neuroscience Institute \\
\\
{\bf Princeton University}, Princeton, NJ  \dotfill 2012--2016\\
{\it Lewis-Sigler Fellow}, Lewis-Sigler Institute for Integrative Genomics\\ 
\\
{\bf Harvard University}, 
Cambridge, MA  \dotfill  2007-2012\\
{\it NSF Graduate Research Fellow}, Program in Biophysics and Department of Physics.\\
\\
{\bf JILA (NIST-University of Colorado)}, Boulder, CO   \dotfill Summers 2005-2006\\
{\it NSF Summer Undergraduate Research Fellow}.\\
\\
{\bf American Association for the Advancement of Science}, Washington, DC  \dotfill Spring 2006\\
{\it Leonard Rieser Fellow}, Center for Science Technology and Security Policy.\\
%	2005-2007	Stanford Daily, Stanford CA.
\\
{\bf Natl.~Telecommunications and Information Administration}, Boulder, CO  \dotfill Summer 2004\\
{\it Researcher}, Institute for Telecommunication Sciences, Theory Division.\\
\\
{\bf National Institute of Standards and Technology}, Boulder, CO  \dotfill Summer 2003\\
{\it Researcher}, Statistics Division.\\
%%%%%%%%%%%%%%%%%%%%%%%%%
%\section{Research Interests}
%\noindent My research focuses on precision measurements in biology using tractable model systems. At all levels, biological organisms exhibit collective phenomena, where large-scale behaviors arise from the action of local players. Members of my lab study the emergence of cellular- and population-level order in prokaryotes by probing the coordinated action of thousands of individual proteins inside a cell and by measuring the dynamics of hundreds of cells simultaneously within a population. Another aspect of our work takes a data-driven approach to studying animal behavior in order to discover deep connections between genetics, neurocircuitry and behavior.
%\noindent My research focuses on developing new, precision imaging and image analysis techniques to study the emergence of large-scale movements from the collective action of individual molecules. By combining new methods for physically probing cells and unique experimental pipelines, I have developed a deep understanding of how cells grow into the correct size and shape, and how cells glide to form large-scale population structure.  %More recently, I have embarked on a major effort to quantify animal movements using statistical techniques. 
%Another aspect of my work takes a data-driven approach to studying animal behavior in order to discover deep connections between genetics, neurocircuitry and behavior.

%%%%%%%%%%%%%%%%%%%%%%%%%

\vspace{-\baselineskip}
\section{Education}
{\bf Ph.D. in Biophysics}, Harvard University, Cambridge, MA \dotfill May 2012

\begin{list1}
\item Thesis Topic: ``Optogenetics and computer vision for \textit{C. elegans} neuroscience and other biophysical applications'' Advisor:  Professor Aravinthan D.T.~Samuel
\end{list1}

{\bf B.S. in Physics},  Stanford University, Stanford, CA \dotfill June 2007\\
{\bf B.A. in Political Science}, Stanford University, Stanford, CA \dotfill June 2007\\

%Interdisciplinary Honors in International Security Studies
Honors in International Security Studies, CISAC, Stanford University, Stanford, CA \dotfill June 2007

\begin{list1}
\item Thesis Topic: ``International scientific engagement for mitigating emerging nuclear security threats'' Advisor: Professor Michael May
\end{list1} 
%%%%%%%%%%%%%%%%%%%%%%%%%
\section{Honors and Awards} 
National Institutes of Health Director's New Innovator Award \dotfill 2019\\
National Science Foundation CAREER Award \dotfill 2019\\
Emerging Leaders in Biosecurity Initiative Fellow, Johns Hopkins,  Center for Health Security \dotfill 2015\\
Simons Investigator, Simons Collaboration on the Global Brain, Simons Foundation \dotfill 2014\\
American Physical Society, Biological Physics Thesis Award: Certificate of Merit \dotfill 2013\\
Lewis-Sigler Fellowship, Princeton University \dotfill 2012--2016\\
Derek C.~Bok Certificate of Distinction in Teaching, Harvard University.\dotfill 2008\\
National Science Foundation Graduate Research Fellowship\dotfill 2007--2011\\
Leonard Rieser Fellowship in Science Tech \& Global Security, Bulletin of the Atomic Scientist\dotfill 2006\\
SPIE International Society for Optical Engineering Scholarship\dotfill 2006\\
American Institute of Physics, Society of Physics Students, Leadership Award\dotfill 2006\\
National Science Foundation, Summer Undergraduate Research Fellowship\dotfill 2005--2006\\
AAAS, Center for Science Technology and Security Policy, Intern of the Year Award.\dotfill 2006\\
Harry Press Journalism Award, Stanford University.\dotfill 2006\\
Boothe Prize for Excellence in Writing, Stanford University\dotfill 2004\\
Robert C. Byrd Academic Merit Scholarship\dotfill 2003\\
Dofflemyer Eagle Scout Scholarship\dotfill 2003\\
Awards for the author's independent research, ``Fractals, Power-Laws and the Weibull Distribution: Mathematically Modeling Crumpled Paper''\dotfill 2003\\
\vspace{-\baselineskip}
\begin{listinline}
\item American Mathematical Society, Karl Menger Award.
\item Office of Naval Research, Naval Science Award.
\item Third Place Team Project, Intel International Science and Engineering Fair 2003.
\item First Place Team Project, Colorado Science and Engineering Fair.
\item Scientific American, Outstanding Achievement in Education.
\end{listinline}
Golden State Governor's Scholarship, State of California.\dotfill 2000\\




%%%%%%%%%%%%%%%%%%%%%%%%%
\section{Departmental or Unit-level Service (AY 2023-2024)} 
Department of Physics:
\begin{listinline}
\item  Chair of Equity, Diversity, and Inclusion (EDI) Advisory Board
\end{listinline}
Princeton Neuroscience Institute:
\begin{listinline}
\item Website Committee Chair;  Neuroscience Grad Admissions Committee
\end{listinline}
Center for Physics of Biological Function:
\begin{listinline}
\item Chair of Seminar Series Committee;  CPBF Fellow Search Committee
\end{listinline}
Biophysics Graduate Program:
\begin{listinline}
\item  Chair Grad Admissions Committee
\end{listinline}



\section{Departmental Service (Previous)}
\begin{listinline}
\item    Chair of the Dicke Fellows Committee (PHY); EDI Committee (PHY); Senior Committee (PHY); Junior Committee (PHY); Rising Stars in Physics Program Committee; Retreat Co-Organizer (NEU); Faculty Search Committee (Ommen-Darling Bioengineering Institute)
\end{listinline}


\section{University Service}
Institutional Biosafety Committee, Princeton University\dotfill 2021--Present\\ 
Undergrad Advisor, Mathey College, Princeton University \dotfill 2020--2023, 2024--present\\
Member, Council of the Princeton University Community\dotfill 2013-2014\\
Senior Staff Committee Member, Lowell House, Harvard College, \dotfill 2010--2012\\ 
Resident Tutor, Lowell House, Harvard College  \dotfill 2009--2012\\
%Editorial Board Member, Stanford Daily, Stanford University \dotfill 2006--2007\\

\section{Professional Service}
%Judge, Delaware Neuroscience Symposium\dotfill 2014\\
Co-organizer, \textit{CeNeuro} conference, to be held in Spain 2026\dotfill 2024-present\\
Co-organizer Physics of Neural Systems sessions, 2025 APS Meeting \dotfill 2024\\
Program Committee member, CoSyNe\dotfill 2019-2022\\
%Founding project leader, Princeton Open Ventilation Monitor Collaboration\dotfill 2020\\
Scientific Program Committee member, International \textit{C. elegans} Conference\dotfill 2019\\
Organizer, Simons Foundation, Workshop on Unbiased Quantification of Behavior\dotfill 2016\\
%Faculty Fellow, Mathey College, Princeton University\dotfill 2015 to present\\
%Invited Participant, NSF Worskshop:~Frontiers for Integrative Study of Animal Behavior\dotfill 2014\\
%Session Chair, \textit{C. elegans} topic mtg: Neuronal Development, Synaptic Function \& Behavior\dotfill 2014\\
%Chair,  Program in Neuroscience Graduate Generals Exam Committee, Princeton University\dotfill 2013\\
%Grant reviewer panelist for National Science Foundation, Integrative Organismal Systems; and National Institutes of Health Somatosensory and Pain Systems Study Section.\\
% NIH Special emphasis panel: Special Topics: Vision Imaging, Bioengineering and Low Vision Technology Development Oct 28 2021
Grant reviewer for funding agencies and foundations including:
\begin{listinline}
\item Agence Nationale de la Recherche (France), European Research Commission (EU),  Israel Science Foundation (Israel),  Medical Research Council (UK),   NASA  (USA), National Institutes of Health (USA), National Science Foundation (USA), NWO (Netherlands), Sir Henry Dale Wellcome Trust (UK), SNSF (Switzerland), W.~M.~Keck Foundation (USA)
\end{listinline}
Scientific content reviewer for peer-reviewed journals including:
\begin{listinline} 
\item \textit{Current Biology,
%Current Opinions in Systems Biology, 
%eLife, %IEEE/ACM, Integrative Biology, Journal of Neuroscience Methods, 
eLife, %Journal of Physical Biology, 
%Journal of Visual Experiments, 
%Nature Communications, 
%Physical Review X,
Nature, Nature Methods, Neuron, %Philosophical Transactions of the Royal Society B, 
Physical Review Letters,  PLOS Biology, PLOS Computational Biology,
%PLoS One, 
PNAS %Scientific Reports
}
\end{listinline}
%Ad-hoc Reviewing Editor: \textit{eLife}


%%%%%%%%%%%%%%%%%%%%%%%%%
\section{Teaching}
Princeton University, {\it Faculty}:\\
PHY 108 Physics for Life Scientists \dotfill Spring 2024\\
NEU 457 (557) Measurement and Analysis of Neural Dynamics \dotfill Spring 2017, 2021, 2023\\
PHY 101 Introductory Physics I \dotfill Fall 2018, 2020-22\\
PHY 103 General Physics I \dotfill Fall 2016, 2019\\
NEU 422 Neural Dynamics of Cognition \dotfill Fall 2017\\
ISC 233-234 An Integrated, Quantitative Intro to the Natural Sciences II, \dotfill 2013--2016\\
ISC 231-232 An Integrated, Quantitative Intro to the Natural Sciences I,  \dotfill 2012--2015\\
Neurotechnologies and Analysis of Neural Datasets, \dotfill Summers 2015--2019\\
CPBF Physics of Life  \dotfill Summers 2018-19, 2022-23\\
\\
Princeton University, {\it Guest Lecturer}:\\
NEU 501,502 Neuroscience: from molecules to systems and behavior \dotfill 2017--2022\\
Warrior Scholar Project, Physics \dotfill 2023\\
SPIA 548, Weapons of Mass Destruction and International Security \dotfill 2017--2019, 2020\\
SPIA 353, Science and Global Security, \dotfill 2015, 2017\\
NEU 301 Cellular Neurobiology \dotfill 2016\\
QCB 551 Intro to Genomics \& Computational Molecular Biology, \dotfill 2014\\	 
%Biophysics and Computations in Neurons and Networks, {\it Assistant Instructor}\dotfill Summer 2013\\
\\
Elsewhere:\\
Stanford University, CS 379C, Computational Models of the Neocortex, {\it Guest Lecturer}\dotfill 2016\\
Marine Biological Laboratory, Woods Hole, Neural Systems \& Behavior, {\it Faculty}\dotfill Summer 2014\\
Harvard University, BIOPHYS 242R, Brain \& Behavior, {\it Guest Lecturer}\dotfill 2013\\ 
Harvard University, MCB 199, Statistical Thermodynamics for Quantitative Biology, {\it T.A.}\dotfill 2008


\section{Advising}
PhD Students (current):
\begin{list1}
\item Pearl Thijssen (PHY), Wayan Gauthey (NEU, joint w/ Murthy), Emily Osborne (PHY), Kevin Chen (NEU, joint w/ Pillow), Sophie Dvali, (PHY), Sandeep Kumar (NEU).
\end{list1}
\vspace{-1\baselineskip}
PhD Students (past):
\begin{list1}
\item  Xinwei Yu (PHY),  Ashley Linder (Neuroscience, joint w/ Shaevitz), Mochi Liu (QCB, joint w/ Shaevitz)
\end{list1}

Undergraduate Students (current):
\begin{list1}
\item   Abdul-Bassit Fijabi (NEU, Senior Thesis), Anna Borodianski (NEU, Junior Project),
Laura Sandoval (NEU, Junior Project) 
\end{list1}
\vspace{-\baselineskip}

Undergraduate Students (past):
\begin{list1}
\item  Andrew Tan (NEU, Senior Thesis), Tori Edington (PHY, Senior Thesis),  Milena Chakraverti-Wuerthwein (PHY, JP and Senior Thesis), John Li (NEU, Senior Thesis), Alicia Castillo (NEU, Senior Thesis), Xiaoting Sun;  David Mazumder (MOL); Kevin Mizes (PHY, Senior Thesis; Treiman Fellow; Sanda \& Jeremiah Lambert '55 Undergraduate Neuroscience Research Award Recipient), Peter Johnson (PHY, Junior Project); Jose Rico Chinchilla; Lukas Novak.
\end{list1}
\vspace{-1\baselineskip}\
\vspace{-1\baselineskip}\


%%%%%%%%%%%%%%%%%%%%%%%%%
\section{Invited Lectures}
%Princeton University, Machine Learning in Physics Symposium \dotfill 2024\\
Harvard University, Department of Physics Colloquium \dotfill 2024\\
Albert Einstein School of Medicine, Department of Neuroscience \dotfill  2023\\
University of Washington, Department of Physiology and Biophysics \dotfill 2023\\
Allen Institute for Neural Dynamics \dotfill 2023\\
Memorial Sloan Kettering Cancer Center, Developmental Biology Seminar  \dotfill 2023\\
APS March Meeting (delivered by Sophie Dvali) \dotfill 2023\\
University of Chicago, Neuroscience Seminar  \dotfill 2023\\
New York Area Worm Meeting, Plenary speaker \dotfill 2023\\
Yale University, Quantitative  Biology Seminar \dotfill 2022\\
Google Research  \dotfill 2022\\
Syracuse University, Department of Physics \dotfill 2022\\
UCSF \dotfill 2022\\
CalTech, Neuroscience Seminar \dotfill 2022\\
Stanford University, Wu Tsai Neurosciences Institute \dotfill 2022\\
Johns Hopkins University, Biology Seminar \dotfill 2022\\
Kavli Institute for Theoretical Physics, Neurophysics of Locomotion Workshop  \dotfill 2022\\
Neuro 2022, Japan Neuroscience Society, Okinawa, Japan \dotfill 2022\\
CoSyNe Workshop, Lisbon, Portugal\dotfill 2022\\ 
Simons Foundation, Simons Collaboration on the Global Brain Annual Meeting \dotfill 2022\\
NSF Workshop: Functional Logic of Neural Circuits, San Juan, PR \dotfill 2022\\
Washington University of St.~Louis, Department of Physics Colloquium \dotfill 2021\\
Society for Neuroscience Short Course, Quantifying Behavior \dotfill 2019\\
Workshop on the Aging Brain, Simons Foundation \dotfill 2019\\
Rockefeller University \dotfill 2019\\
National Institutes of Health BRAIN Initiative Investigators Meeting \dotfill 2019\\
Vanderbilt University, Department of Physics and Astronomy Colloquium \dotfill 2019\\
Columbia University, Center for Theoretical Neuroscience \dotfill 2018\\
SAND8, Statistical Analysis of Neuronal Data, Keynote Lecturer \dotfill  2017\\
Rowen University School of Osteopathic Medicine, Department of Cell Biology \dotfill 2017\\
APS March Meeting, Patterns \& Control in Animal Behavior \dotfill  2017\\
CUNY, The Graduate Center, Initiative for the Theoretical Sciences\dotfill  2016\\
Cornell University, NBB, Perry Gilbert Lecture, Invited by Grad Students\dotfill  2016\\
ICFO, Institute of Photonic Sciences, Light for Health Seminar \dotfill  2016 \\
Simons Foundation, Simons Collaboration on the Global Brain Annual Meeting \dotfill 2016 \\ 
Frontiers in Applied \& Computational Mathematics\dotfill 2016\\
Mid-Atlantic Society for Developmental Biology Regional Meeting \dotfill  2016\\
Yale University School of Medicine, Department of Neuroscience Seminar \dotfill  2016\\
Princeton University, Princeton Neuroscience Institute Seminar \dotfill 2016\\
Yale University, Dept.~of Molecular Cellular \& Developmental Biology Seminar \dotfill 2016\\
Google, Inc. \dotfill 2016\\
Stanford University School of Medicine, Department of Neurobiology Seminar \dotfill  2016\\
Ludwig Maximilians Universitat, Munchen, Center for Nanoscience Colloquium \dotfill 2015\\
Northeastern University, Center for Complex Network Research \dotfill 2015 \\ 
Princeton University, Woodrow Wilson School, Science and Global Security Seminar \dotfill 2015\\
Simons Foundation, Simons Collaboration on the Global Brain Annual Meeting \dotfill 2015\\
Rockefeller University, Center for Studies in Physics and Biology Seminar \dotfill 2015\\
 Stanford University, Stanford Neurosciences Institute \& Department of Bioengineering \dotfill 2015\\
 New York University, Center for Soft Matter Research \dotfill 2015\\
 Delaware Center for Neuroscience Research  \dotfill 2014\\
  Brandeis University, Computational \& Systems Neuroscience Journal Club \dotfill  2014\\
 Columbia University, Grossman Center, Quantifying Structure in Large Neural Datasets \dotfill 2014\\
 \textit{C. elegans} topic meeting: Neuronal Development, Synaptic Function
\& Behavior \dotfill 2014\\ 
 Rutgers University, Multi Group Worm Meeting  \dotfill 2013 \\
 INSERM, University of Paris Descartes, Optics and Photonics Seminar \dotfill 2012\\
 Princeton University, Lewis-Sigler Institute for Integrative Genomics \dotfill 2011 \\
 Rutgers University,  Molecular Biology and Biochemistry \dotfill 2010\\
 Harvard University, Rowland Institute  \dotfill 2010 \\
%\vspace{-2\baselineskip}\   %Omit this when the page break no longer falls here
%%%%%%%%%%%%%%%%%%%%%%%%%




\section{Manuscripts undergoing peer review}
\begin{bibsection}
 
\item Matthew S Creamer, Andrew M Leifer, Jonathan W Pillow, ``Bridging the gap between the connectome and whole-brain activity in C. elegans,'' bioRxiv, 2024.09.22.614271(2024).

\item Kevin S. Chen, Anuj K. Sharma, Jonathan W. Pillow, Andrew M. Leifer, ``Olfactory learning alters navigation strategies and behavioral variability in C. elegans,'' arXiv:2311.07117 [q-bio.NC]; 13 Nov (2023). 
\end{bibsection}

\section{Peer-reviewed Publications}
\begin{bibsection}

\item Jing Huo, Tianqi Xu, Qi Liu, Mahiber Polat, Sandeep Kumar, Xiaoqian Zhang, Andrew M. Leifer, Quan Wen, ``Hierarchical behavior control by a single class of interneurons,'' \textit{PNAS}, 121 (47) e2410789121, 12 November (2024).

\item Sandeep Kumar, Anuj K Sharma, Andrew M Leifer, ``An inhibitory acetylcholine receptor gates context dependent mechanosensory processing in C. elegans,'' \textit{iScience}, Volume 27, Issue 10, 18 October (2024).

\item Anuj Kumar Sharma, Francesco Randi, Sandeep Kumar, Sophie Dvali, Andrew M.~Leifer, ``TWISP: A Transgenic Worm for Interrogating Signal Propagation in C. elegans.'' \textit{Genetics}, Vol 227, Issue 3, July (2024).


\item Wayan Gauthey, Francesco Randi*, Anuj K.~Sharma,* Sandeep Kumar, Andrew M.~Leifer, ``Light evokes stereotyped global brain dynamics in C. elegans.'' \textit{Current Biology}, Vol 34, Issue 1, Pages R14-R15, 8 January (2024).

\item Francesco Randi, Anuj K.~Sharma, Sophie Dvali, Andrew M.~Leifer, ``Neural signal propagation atlas of C. elegans.'' \textit{Nature},  623, 406–414  (2023).

\item Sandeep Kumar,  Anuj K.~Sharma, Andrew Tran, Mochi Liu, Andrew M.~Leifer, ``Inhibitory motor signals gate mechanosensory processing in C. elegans'' \textit{PLOS Biology},  (9): e3002280 (2023).


\item Kevin S.~Chen*, Rui Wu*, Marc H.~Gershow, and Andrew M.~Leifer. ``Continuous odor profile monitoring to study olfactory navigation in small animals.'' \textit{eLife}, 12:e85910 , 25 July (2023).

\item Matthew S.~Creamer, Kevin S.~Chen, Andrew M.~Leifer, Jonathan W.~Pillow, ``Correcting motion induced fluorescence artifacts in two-channel neural imaging.'' \textit{PLOS Computational Biology},  18(9): e1010421  Sept 28 (2022)

\item Princeton Open Ventilation Monitor Collaboration, Philippe Bourrianne, Stanley Chidzik, Daniel J Cohen, Peter Elmer, Thomas Hallowell, Todd J Kilbaugh, David Lange, Andrew M.~Leifer, Daniel R. Marlow, Peter D. Meyers, Edna Normand, Janine Nunes, Myungchul Oh, Lyman Page, Talmo Pereira, Jim Pivarski, Henry Schreiner, Howard A Stone, David W Tank, Stephan Thiberge, Christopher Tully.  Inexpensive multi-patient respiratory monitoring system for helmet ventilation during COVID-19 pandemic. \textit{ASME Journal of Medical Devices.}  Mar 16(1): 011003 (2022).

\item Mochi Liu, Sandeep Kumar, Anuj K Sharma, Andrew M. Leifer. ``A high-throughput method to deliver targeted optogenetic stimulation to moving C. elegans populations.'' \textit{PLOS Biology} 20(1): e3001524. (2022)



\item Anne E. Urai, Brent Doiron, Andrew M. Leifer, Anne K. Churchland. ``Large-scale neural recordings call for new insights to link brain and behavior.'' \textit{Nature Neuroscience}, 3 January (2022). [Invited Review]


\item Kelsey M.~Hallinen*, Ross Dempsey*, Monika Scholz*, Xinwei Yu, Ashley N Linder, Francesco Randi, Anuj K Sharma, Joshua W. Shaevitz and Andrew M Leifer, ``Decoding locomotion from population neural activity in moving C. elegans.'' \emph{eLife}, 10:e66135, 29 July (2021).

\item Xinwei Yu, Matthew S.~Creamer, Francesco Randi, Anuj K.~Sharma, Scott W.~Linderman, Andrew M.~Leifer, ``Fast deep neural correspondence for  tracking and identifying neurons in C.~elegans using semi-synthetic training.'' \emph{eLife}, 10:e66410, 14 July (2021).

\item Francesco Randi and Andrew M.~Leifer, ``Nonequilibrium Green's functions for functional connectivity in the brain.'' \emph{Phys Rev Lett}, \textbf{126}, 118102 (2021).

\item Francesco Randi and Andrew M.~ Leifer. ``Measuring and modeling whole-brain neural dynamics in Caenorhabditis elegans.'' \emph{Current Opinion in Neurobiology}. Vol 65, Pages 157-167 (2020). [Invited Review]

\item Robert Datta, David Anderson, Kristen Branson, Pietro Perona, and Andrew Leifer, ``Computational neuroethology: a call to action.'' \emph{Neuron}, 104:1, (2019). [Review]

\item Xiaowen Chen, Francesco Randi, Andrew M Leifer and William Bialek, ``Searching for collective behavior in a small brain.'' \emph{Phys Rev E} \textbf{99}, 052418 (2019).

\item Mochi Liu, Anuj K.~Sharma, Joshua W.~Shaevitz, Andrew M.~Leifer, ``Temporal processing and context dependency in \textit{C. elegans} mechanosensation.'' \emph{eLife}, 7:e36419 (2018).   

\item Jeffrey Nguyen,  Ashley N.~Linder, George Plummer, Joshua
W.~Shaevitz, Andrew M.~Leifer, ``Automatically tracking neurons in a moving and deforming brain'' \emph{Plos Computational Biology}, 13(5): e1005517 (2017).
	
\item Jeffrey Nguyen$^{*}$, Frederick B.~Shipley$^{*}$, Ashley N.~Linder, George Plummer, Mochi Liu, Sagar U. Setru, Joshua
W.~Shaevitz, Andrew M.~Leifer, ``Whole-brain calcium imaging with cellular resolution in freely behaving \textit{Caenorhabditis elegans}.'' \emph{Proceedings of the National Academy of Sciences,}  vol.~113 no.~8, E1074-E1081 (2016). 

\item Frederick B.~Shipley, Christopher M.~Clark,  Mark J.~Alkema, Andrew M.~Leifer, ``Simultaneous optogenetic stimulation and calcium imaging in freely moving \textit{C. elegans}.'' \emph{Frontiers in Neural Circuits} 8:28 (2014).
	
\item Steven J.~Husson, Alexander Gottschalk, Andrew M.~Leifer, ``Optogenetic manipulation of neural activity in C. elegans: from synapse to circuits and behavior'' \emph{Journal of Biology of the Cell}, 105, 1--16 (2013).  [Invited Review]

\item Jamie L.~Donnelly, Christpoher M.~Clark, Andrew M.~Leifer, Marian Haburacak, Jennifer K.~Pirri, Michael M.~Francis, Aravinthan D.~T.~Samuel, and Mark J.~Alkema. ``Monoaminergic orchestration of motorprograms in a complex behavior in C. elegans.'' \emph{PLoS Biology}  11(4): e1001529 (2013).	

\item Quan Wen, Michelle Po, Elizabeth Hulme, Sway Chen, Xinyu Liu, Sen Wai Kwok, Marc Gershow,  Andrew M.~Leifer, Victoria Butler, Christopher Fang-Yen, Taizo Kawano, William R.~Schafer, George Whitesides, Matthieu Wyart, Dmitri Chklovskii, Mei Zhen, Aravinthan D T Samuel, ``Proprioceptive coupling within motor neurons drives \emph{C. elegans} forward locomotion.'' \emph{Neuron}, 76, 750--761 (2012).

\item Chenxiang Lin, Ralf Jungmann, Andrew M.~Leifer, Chao Li, Daniel Levner, Geroge M.~Church, William M.~Shih, Peng Yin. ``Sub-micrometer geometrically encoded fluorescent barcodes self-assembled from DNA.'' \emph{Nature Chemistry}, 4, 832--839 (2012).

\item Andrew M. Leifer$^{*}$, Christopher Fang-Yen$^{*}$, Marc Gershow, Mark Alkema, Aravinthan D.T.~Samuel, ``Optogenetic manipulation of neural activity in freely moving \emph{Caenorhabditis elegans},'' \emph{Nature Methods}, 8(2), p.147â--152 (2011) . 

\item Kevin J.~Coakley, David S.~Simons, Andrew M.~Leifer. ``Secondary Ion Mass Spectrometry Measurements of Isotopic Ratios: Correction for Time Varying Count Rate.'' \emph{International Journal of Mass Spectrometry}, 204, 107--120 (2005).

\end{bibsection}


\section{Books Under Contract}
\begin{bibsection}
\item Ross Dempsey and Andrew M.~Leifer. \textit{Undergraduate Physics for Graduate Students}. Princeton University Press. Expected 2026. 
\end{bibsection}
	
%%%%%%%%%%%%%%%%%%%%%%%%%
%\section{Current and Past Funding}
\section{Active or Awarded Grants}


7/2024--7/2026, Simons Foundation, Simons Collaboration on the Global Brain, SCGB \#3196-03 (PI Leifer; spokesperson PI is Zimmer) \\
``Neuromodulatory interactions the control of long-time scale behaviors''\\
Total Direct \& Indirect Costs: \$436,400\\

6/2024--6/2025, Princeton Neuroscience Institute Innovation Award, Princeton University\\ (PIs: Leifer \& Shaevitz)\\
``Spatiotemporal dynamics of neuropeptide signaling in the brain''\\
Total Direct \& Indirect Costs: \$200,000\\

5/2024--4/2026, Eric \& Wendy Schmidt Technology Fund, Princeton University\\ (PIs: Leifer \& Murthy)\\
``New Technology for Brain-Wide Functional Connectivity Measurements at Cellular Resolution''\\
Total Direct \& Indirect Costs: \$1,100,000\\

5/2024--4/2026, Ommen-Darling Bioengineering Institute, Princeton University\\ (PIs: Leifer \& Murthy)\\
``Technology for measuring neural signal propagation at brain-scale''\\
Total Direct \& Indirect Costs: \$140,000\\

9/18/2019--8/31/2024  National Institute of Health, 1DP2NS116768, (PI: Leifer)\\
``Probing brain-wide functional connectivity during behavior.''\\ 
Total Direct \& Indirect Costs: \$2,430,000\\\ %Andy double check the exa


6/2019--5/2024 National Science Foundation, 1845137, (PI: Leifer) \\
``CAREER: Neural mechanisms of flexible sensorimotor processing in C. elegans'' \\
Total Direct \& Indirect Costs: \$800,000 \\


7/2017--7/2024, Simons Foundation, Simons Collaboration on the Global Brain, SCGB \#543003 (PI Leifer; spoksepserson PI is Zimmer) \\
``Neural Dynamics of a Multi-timescale Social Behavior''\\
Total Direct \& Indirect Costs: \$1,080,000\\

10/1/2017--9/30/2025 National Science Foundation, PHY-1734030 (PI: Bialek, co-PI: Shaevitz, named investigator: Leifer)\\
``Physics Frontier Center: Center for the Physics of Biological Function'' \\
Total Direct \& Indirect Costs: \$14,700,000




\section{Completed Grants}
5/15/2020--4/30/2021 National Science Foundation, PHY-2031509, (co-PI: Leifer; PI: Elmer)\\
RAPID: Open Research Infrastructure for COVID-19 Ventilator Data\\
Total Direct \& Indirect Costs: \$200,000\\\ 


%Award ID 324285
%This work was supported by a grant from the Simons Foundation (SCGB#324285, A.L.)
7/2014--7/2017, Simons Foundation, Simons Collaboration on the Global Brain, SCGB  (PI: Leifer) \\
``Whole brain calcium imaging in freely behaving nematodes''\\
Total Direct \& Indirect Costs: \$320,000\\

9/2017--8/2019  National Institute of Health, 1R21NS101629, (PI: Murray, U Penn)\\
``Multicolor labeling for cell identification in the \textit{C. elegans} nervous system''\\ 
Total Direct \& Indirect Costs: \$500,000  (\$250,000 to Leifer) \\\

9/2014--8/2016, Princeton University, Inaugural Dean's Innovation Fund for New Ideas in the Natural Sciences (co-PI with Shaevitz)\\
``All-neuron I/O in freely behaving animals''\\
Total Direct Costs: \$200,000 (\$100,000 to Leifer) \\


\end{document}




