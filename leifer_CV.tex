% Andrew Leifer's CV
% Based on Joshua Shaevitz's Latex CV template of unknown origin

%%%%%%%%%%%%%%%%%%%%%%%%%%%% Document Setup %%%%%%%%%%%%%
\documentclass[11pt]{article}
\usepackage[paper=letterpaper,margin=1in]{geometry}

%% More layout: Get rid of indenting throughout entire document
\setlength{\parindent}{0in}

%% This gives us fun enumeration environments. compactitem will be nice.
\usepackage{paralist}

% This is a helpful package that puts math inside length specifications
\usepackage{calc}

% Simpler bibsection for CV sections
% (thanks to natbib for inspiration)
\makeatletter
\newlength{\bibhang}
\setlength{\bibhang}{2em}
\newlength{\bibsep}
\newcounter{Lcount}
 {\@listi \global\bibsep\itemsep \global\advance\bibsep by\parsep}
\newenvironment{bibsection}%
        {\begin{list}{\arabic{Lcount}.}{%
       \usecounter{Lcount}%
       \setlength\labelwidth{-0.5em}%
       \setlength{\leftmargin}{\bibhang}%
       \setlength{\itemindent}{-\leftmargin}%
       \setlength{\itemsep}{\bibsep}%
       \setlength{\parsep}{\z@}%
        \setlength{\partopsep}{0pt}%
        \setlength{\topsep}{0pt}}}
        {\end{list}\vspace{-.6\baselineskip}}
\makeatother

\usepackage{lastpage} 
\usepackage{fancyhdr}
\usepackage{hyperref}

\pagestyle{fancy}
%\pagestyle{empty}      % Uncomment this to get rid of page numbers
\fancyhf{}\renewcommand{\headrulewidth}{0pt}
\fancyfootoffset{\marginparsep+\marginparwidth}
\newlength{\footpageshift}
\setlength{\footpageshift}
          {0.5\textwidth+0.5\marginparsep+0.5\marginparwidth-2in}
%\lfoot{\hspace{\footpageshift}%
%       \parbox{4in}{\, \hfill %
%                    \arabic{page} of \protect\pageref{LastPage} % +LP
%                    \arabic{page}                               % -LP
%                    \hfill \,}}
\lhead{Andrew~M.~Leifer}
\rhead{Curriculum Vitae}
\fancyfoot[CO,CE]{ \arabic{page} of \protect\pageref{LastPage}}
\renewcommand{\headrulewidth}{.1pt}


%%%%%%%%%%%%%%%%
%Lists and Section headings
%%%%%%%%%%%%%%%
\newenvironment{list1}{
%\vspace{-\baselineskip}
  \begin{list}{}{
      \setlength{\itemsep}{0in}
      \setlength{\parsep}{0in} \setlength{\parskip}{0in}
      \setlength{\topsep}{0in} \setlength{\partopsep}{0in} 
      \setlength{\leftmargin}{0.2in}\setlength{\rightmargin}{0.2in}}}{\end{list}\vspace{\baselineskip}}

\newenvironment{listinline}{
	%\vspace{-\baselineskip}
	  \begin{list}{}{
	      \setlength{\itemsep}{0in}
	      \setlength{\parsep}{0in} \setlength{\parskip}{0in}
	      \setlength{\topsep}{0in} \setlength{\partopsep}{0in} 
	      \setlength{\leftmargin}{0.2in}\setlength{\rightmargin}{0.2in}}}{\end{list}}


\makeatletter
\renewcommand{\section}{\@startsection
{section}%                   % the name
{1}%                         % the level
{0mm}%                       % the indent
{2\baselineskip}%            % the before skip
{0.5\baselineskip}%          % the after skip
{\normalfont\bf\MakeUppercase}} % the style
\makeatother
\setcounter{secnumdepth}{-1} 

%%%%%%%%%%%%%%%%%%%%%%%% End Document Setup %%%%%%%%%%%%%%%%%%%%%%%%%%%%

\begin{document}
\thispagestyle{empty}

\begin{center}
\vspace{-\baselineskip}
{\large {\bf CURRICULUM VITAE
\vskip 0.1in
Andrew M.~Leifer}

Lewis-Sigler Fellow and Lecturer of Physics

Princeton University}
\end{center}

%%%%%%%%%%%%%%%%%%%%%%%%%
\section{Contact Information}

170 Carl Icahn Laboratories \hfill Phone: (609) 258-2973 \\
Lewis-Sigler Institute \hfill leifer@princeton.edu\\
Princeton, NJ 08544 \hfill \url{http://leiferlab.princeton.edu}\\
%%%%%%%%%%%%%%%%%%%%%%%%%

\vspace{-\baselineskip}
\section{Professional Experience}

{\bf Princeton University}, Princeton, NJ  \dotfill 2012--present\\
{\it Lewis-Sigler Fellow}, Lewis-Sigler Institute for Integrative Genomics\\ 
{\it Lecturer}, Department of Physics.\\
\\
{\bf Harvard University}, 
Cambridge, MA  \dotfill  2007-2012\\
{\it NSF Graduate Research Fellow}, Program in Biophysics and Department of Physics.\\
\\
{\bf JILA (NIST-University of Colorado)}, Boulder, CO   \dotfill Summers 2005-2006\\
{\it NSF Summer Undergraduate Research Fellow}.\\
\\
{\bf American Association for the Advancement of Science}, Washington, DC  \dotfill Spring 2006\\
{\it Leonard Reiser Fellow}, Center for Science Technology and Security Policy.\\
%	2005-2007	Stanford Daily, Stanford CA.
\\
{\bf Natl.~Telecommunications and Information Administration}, Boulder, CO  \dotfill Summer 2004\\
{\it Researcher}, Institute for Telecommunication Sciences, Theory Division.\\
\\
{\bf National Institute of Standards and Technology}, Boulder, CO  \dotfill Summer 2003\\
{\it Researcher}, Statistics Division.\\
%%%%%%%%%%%%%%%%%%%%%%%%%
%\section{Research Interests}
%\noindent My research focuses on precision measurements in biology using tractable model systems. At all levels, biological organisms exhibit collective phenomena, where large-scale behaviors arise from the action of local players. Members of my lab study the emergence of cellular- and population-level order in prokaryotes by probing the coordinated action of thousands of individual proteins inside a cell and by measuring the dynamics of hundreds of cells simultaneously within a population. Another aspect of our work takes a data-driven approach to studying animal behavior in order to discover deep connections between genetics, neurocircuitry and behavior.
%\noindent My research focuses on developing new, precision imaging and image analysis techniques to study the emergence of large-scale movements from the collective action of individual molecules. By combining new methods for physically probing cells and unique experimental pipelines, I have developed a deep understanding of how cells grow into the correct size and shape, and how cells glide to form large-scale population structure.  %More recently, I have embarked on a major effort to quantify animal movements using statistical techniques. 
%Another aspect of my work takes a data-driven approach to studying animal behavior in order to discover deep connections between genetics, neurocircuitry and behavior.

%%%%%%%%%%%%%%%%%%%%%%%%%

\vspace{-\baselineskip}
\section{Education}
{\bf Ph.D. in Biophysics}, Harvard University, Cambridge, MA \dotfill May 2012

\begin{list1}
\item Thesis Topic: ``Optogenetics and computer vision for \textit{C. elegans} Neuroscience and Other Biophysical Applications'' Advisor:  Professor Aravinthan D.T.~Samuel
\end{list1}

{\bf B.S. in Physics},  Stanford University, Stanford, CA \dotfill June 2007\\
{\bf B.A. in Political Science}, Stanford University, Stanford, CA \dotfill June 2007\\

%Interdisciplinary Honors in International Security Studies
Honors in International Security Studies, Stanford University, Stanford, CA \dotfill June 2007

\begin{list1}
\item Thesis Topic: ``International scientific engagement for mitigating emerging nuclear security threats'' Advisor: Professor Michael May
\end{list1} 
%%%%%%%%%%%%%%%%%%%%%%%%%
\section{Honors and Awards} 
Emerging Leaders in Biosecurity Initiative Fellowship, UPMC Center for Health Security\dotfill 2015\\
American Physical Society, Biological Physics Thesis Award, Certificate of Merit \dotfill 2013\\
%Lewis-Sigler Fellowship, Princeton University \dotfill 2012--Present\\
National Science Foundation Graduate Research Fellowship\dotfill 2007--2011\\
Derek C.~Bok Certificate of Distinction in Teaching, Harvard University.\dotfill 2008\\
%Leonard M. Rieser Fellowship
Rieser Fellowship in Science Technology and Global Security, Bulletin of the Atomic Scientist\dotfill 2006\\
SPIE International Society for Optical Engineering Scholarship\dotfill 2006\\
American Institute of Physics, Society of Physics Students, Leadership Award\dotfill 2006\\
National Science Foundation, Summer Undergraduate Research Fellowship\dotfill 2005--2006\\
AAAS, Center for Science Technology and Security Policy, Intern of the Year Award.\dotfill 2006\\
Harry Press Journalism Award, Stanford University.\dotfill 2006\\
Boothe Prize for Excellence in Writing, Stanford University\dotfill 2004\\
Robert C. Byrd Academic Merit Scholarship\dotfill 2003\\
Dofflemyer Eagle Scout Scholarship\dotfill 2003\\
Awards for the author's independent research, ``Fractals, Power-Laws and the Weibull Distribution: Mathematically Modeling Crumpled Paper''\dotfill 2003\\
\vspace{-\baselineskip}
\begin{listinline}
\item American Mathematical Society, Karl Menger Award.
\item Office of Naval Research, Naval Science Award.
\item Third Place Team Project, Intel International Science and Engineering Fair 2003.
\item First Place Team Project, Colorado Science and Engineering Fair.
\item Scientific American, Outstanding Achievement in Education.
\end{listinline}
Golden State Governor's Scholarship, State of California.\dotfill 2000\\




%%%%%%%%%%%%%%%%%%%%%%%%%
\section{Service} 
%Judge, Delaware Neuroscience Symposium\dotfill 2014\\
Invited Participant, NSF Worskshop:~Frontiers for Integrative Study of Animal Behavior\dotfill 2014\\
Session Chair, \textit{C. elegans} topic mtg: Neuronal Development, Synaptic Function \& Behavior\dotfill 2014\\
Member, Council of the Princeton University Community\dotfill 2013-2014\\
Chair,  Program in Neuroscience Graduate Generals Exam Committee, Princeton University\dotfill 2013\\
Senior Staff Committee Member, Lowell House, Harvard College, \dotfill 2010--2012\\ 
Resident Tutor, Lowell House, Harvard College  \dotfill 2009--2012\\
Editorial Board Member, Stanford Daily, Stanford University \dotfill 2006-2007\\
Scientific content reviewer for peer-reviewed journals including:
\begin{listinline}
\item Nature Communications, Journal of Visual Experiments and PLoS One
\end{listinline}
Reviewer or panelist for funding agencies including:
\begin{listinline}
\item National Science Foundation, Division of Integrative Organismal Systems; W.~M.~Keck Foundation; NASA Postdoctoral Program; Sir Henry Dale Wellcome Trust; European Research Commision
\end{listinline}
Content reviewer for conferences including:
\begin{listinline}
\item CoSyNe
\end{listinline}


%%%%%%%%%%%%%%%%%%%%%%%%%
\section{Teaching}
Princeton University:\\
ISC 231-232 An Integrated, Quantitative Intro to the Natural Sciences, {\it Faculty}\dotfill 2012--2014\\
ISC 233-234 An Integrated, Quantitative Intro to the Natural Sciences II, {\it Faculty}\dotfill 2013--2015\\
QCB 551 Intro to Genomics \& Computational Molecular Biology, {\it Guest Lecturer}\dotfill 2014\\	 
Biophysics and Computations in Neurons and Networks, {\it Assistant Instructor}\dotfill Summer 2013\\
\\
Marine Biological Laboratory, Woods Hole:\\
Neural Systems and Behavior, {\it Faculty}\dotfill Summer 2014\\
\\
Harvard University:\\
BIOPHYS 242R, Special Topics in Biophysics: Brain and Behavior, {\it Guest Lecturer}\dotfill 2013\\ 
MCB 199, Statistical Thermodynamics for Quantitative Biology, {\it Teaching Assistant}\dotfill 2008


\section{Advising}
Current PhD Students (jointly advised with Prof.~Joshua Shaevitz):
\begin{list1}
\item Ashley Linder, Program in Neuroscience 
\item Mochi Liu, Quantitative and Computational Biology
\end{list1}
\vspace{-1\baselineskip}
Current Undergraduate Students:
\begin{list1}
\item David Mazumder, Department of Molecular Biology
\item Kevin Mizes, Department of Physics, Treiman Fellow, Sanda \& Jeremiah Lambert '55 Undergraduate Neuroscience Research Award Recipient
\item Jose Rico Chinchilla 
\item Lukas Novak 
\end{list1}
\vspace{-\baselineskip}

Past Undergraduate Students:
\begin{list1}
\item Peter Johnson, Department of Physics, Junior Project
\end{list1}
\vspace{-1\baselineskip}\
\vspace{-1\baselineskip}\


%%%%%%%%%%%%%%%%%%%%%%%%%
\section{Invited Lectures}
Ludwig Maximilians Universitat, Munchen, Center for Nanoscience Colloqium \dotfill expected 2015\\
Princeton University, Princeton Neurosciences Institute, Annual Retreat \dotfill 2015\\
Rockefeller University, Center for Studies in Physics and Biology Seminar \dotfill 2015\\
 Stanford University, Stanford Neurosciences Institute \& Department of Bioengineering \dotfill 2015\\
 New York University, Center for Soft Matter Research \dotfill 2015\\
 Delaware Center for Neuroscience Research  \dotfill 2014\\
 Brandeis University, Computational \& Systems Neuroscience Journal Club \dotfill  2014\\
 Columbia University, Grossman Center, Quantifying Structure in Large Neural Datasets \dotfill 2014\\
 \textit{C. elegans} topic meeting: Neuronal Development, Synaptic Function
\& Behavior \dotfill 2014\\ 
 Rutgers University, Multi Group Worm Meeting  \dotfill 2013 \\
 INSERM, University of Paris Descartes, Optics and Photonics Seminar \dotfill 2012\\
 Princeton University, Lewis-Sigler Institute for Integrative Genomics \dotfill 2011 \\
 Rutgers University,  Molecular Biology and Biochemistry \dotfill 2010\\
 Harvard University, Rowland Institute  \dotfill 2010 \\



%%%%%%%%%%%%%%%%%%%%%%%%%
\section{Peer-reviewed Publications}

\begin{bibsection}
	
\item Frederick B.~Shipley, Christopher M.~Clark,  Mark J.~Alkema, \textbf{Andrew M.~Leifer}, ``Simultaneous optogenetic stimulation and calcium imaging in freely moving \textit{C. elegans}.'' \emph{Frontiers in Neural Circuits} 8:28 (2014).
	
\item Steven J.~Husson, Alexander Gottschalk, \textbf{Andrew M.~Leifer}, ``Optogenetic manipulation of neural activity in C. elegans: from synapse to circuits and behavior'' \emph{Journal of Biology of the Cell}, 105, 1--16 (2013).  \textbf{Invited review.}

\item Jamie L.~Donnelly, Christpoher M.~Clark, \textbf{Andrew M.~Leifer}, Marian Haburacak, Jennifer K.~Pirri, Michael M.~Francis, Aravinthan D.~T.~Samuel, and Mark J.~Alkema. ``Monoaminergic orchestration of motorprograms in a complex behavior in C. elegans.'' \emph{PLoS Biology}  11(4): e1001529 (2013).	

\item Quan Wen, Michelle Po, Elizabeth Hulme, Sway Chen, Xinyu Liu, Sen Wai Kwok, Marc Gershow,  \textbf{Andrew M.~Leifer}, Victoria Butler, Christopher Fang-Yen, Taizo Kawano, William R.~Schafer, George Whitesides, Matthieu Wyart, Dmitri Chklovskii, Mei Zhen, Aravinthan D T Samuel, ``Proprioceptive coupling within motor neurons drives \emph{C. elegans} forward locomotion.'' \emph{Neuron}, 76, 750--761 (2012).

\item Chenxiang Lin, Ralf Jungmann, \textbf{Andrew M.~Leifer}, Chao Li, Daniel Levner, Geroge M.~Church, William M.~Shih, Peng Yin. ``Sub-micrometer geometrically encoded fluorescent barcodes self-assembled from DNA.'' \emph{Nature Chemistry}, 4, 832--839 (2012).

\item \textbf{Andrew M. Leifer}$^{*}$, Christopher Fang-Yen$^{*}$, Marc Gershow, Mark Alkema, Aravinthan D.T.~Samuel, ``Optogenetic manipulation of neural activity in freely moving \emph{Caenorhabditis elegans},'' \emph{Nature Methods}, 8(2), p.147�--152 (2011) . 

\item Kevin J.~Coakley, David S.~Simons, \textbf{Andrew M.~Leifer}. ``Secondary Ion Mass Spectrometry Measurements of Isotopic Ratios: Correction for Time Varying Count Rate.'' \emph{International Journal of Mass Spectrometry}, 204, 107--120 (2005).

\end{bibsection}

\section{Manuscripts pre-review}
\begin{bibsection}
\item Jeffrey Nguyen$^{*}$, Frederick B.~Shipley$^{*}$, Ashley N.~Linder, George Plummer, Joshua W.~Shaevitz, \textbf{Andrew M.~Leifer}, ``Whole-brain calcium imaging with cellular resolution in freely behaving \textit{C. elegans}.'' arXiv:1501.03463.


%\item Christopher M.~Clark$^{*}$, \textbf{Andrew M.~Leifer}$^{*}$, Ni Ji, Jeremey Florman, Kevin Mizes, Aravinthan D.T.~Samuel, Mark J.~Alkema, ``Synaptic chain model for an escape response motor sequence.'' (in prep for resubmission).

\end{bibsection}



%%%%%%%%%%%%%%%%%%%%%%%%%
%\section{Current and Past Funding}
\section{Active Grants}
07/2014--07/2017, Simons Collaboration on the Global Brain Research Award (PI) \\
``Whole brain calcium imaging in freely behaving nematodes''\\
Annual Direct Costs: \$80,000\\
Total Direct Costs: \$240,000\\
%Award ID 324285
%This work was supported by a grant from the Simons Foundation (SCGB#324285, A.L.)

09/2014--08/2016, Inaugural Dean's Innovation Fund for New Ideas in the Natural Sciences (co-PI with Shaevitz)\\
``All-neuron I/O in freely behaving animals''\\
Annual Direct Costs: \$100,000\\
Total Direct Costs: \$200,000\\

%06/2012--06/2017, Lewis-Sigler Fellowship (PI, start-up funds)\\
%``Optical neurophysiology of behavior''\\
%Annual Direct Costs: \$200,000\\
%Total Direct Costs: \$1,000,000




%\textbf{Completed Grants}\\


\end{document}




